\chapter{Basys 2 Programming}

\section{Passthrough}

We are going to begin with a simple project to turn on LEDs when the switch under them is on.  There are eight switches (called sw$\langle 7\rangle$ \ldots sw$\langle 0\rangle$), and eight LEDs  (called Led$\langle 7\rangle$ \ldots Led$\langle 0\rangle$).  Our first easy part will be to assign the LEDs to be identical to the switches, see Code~\ref{code:passthroughsimple}.

\Code{Verilog code for pass-through}{code:passthroughsimple}{pass_through_simple.v}{verilog}

This code simply passes the switches through to the LEDs, but it demonstrates the \textbf{assign} statement, which is one of our basic ways of designing combinational circuits.  The other important thing is to have a user constraints file (UCF).  An example is below.

\Code{User Constraints File}{code:UCFpassthroughsimple}{Basys2_100_250General_a.ucf}{tcl}

\section{Making a Counter for the Seven Segment Display}

Now we want to drive the seven segment display.  All the cathodes of the LEDs in the same position are hooked together.  So for example, all four of the top LEDs' cathodes are hooked together, as are all the cathodes of the upper right, and so on.  The top LED is called A or seg$\langle 0\rangle$, the letters and numbers proceed clockwise, and the middle segment is G or seg$\langle 6\rangle$.  Each digit also has a decimal point, which is called dp, and their cathodes are also connected.   The FPGA does sink the current in this case (\emph{Why can it sink what it couldn't source?}), so a zero turns the LED on if the anode was also sourced.

To drive the seven segment display we will then need to convert from a 4 bit binary number to the seven segment display cathode pattern.  We will leave the decimal point off.  This can be seen in Code~\ref{code:sseg}.

\Code{Verilog code for converting 4 bit binary to sseg.}{code:sseg}{sseg_driver.v}{verilog}

All the anodes for a digit are hooked together, so since there are four digits there are four anode lines (called an$\langle 3\rangle$ \ldots an$\langle 0\rangle$).  The output of the fpga is not strong enough to supply all the current for the LEDs, so a pnp transistor is used to switch power (\emph{Why does a pnp switch power better, and an npn switch ground better?}).  This means a zero, is needed to turn the pnp on and a 1 will turn it off.

This slicing allows control of 32 LEDs with 12 wires and one fourth can be on simultaneously.  Since each LED will be off three fourths of the time, we need to switch them faster than the eye can see, and since the diode's pn junction holds charge for a bit (discharges like a capacitor, why?) and thus continues to glow, and the eye continues to see light for a bit.  The Basys 2 board has a 50MHz clock, so we need to count to about a million, which is 20 bits.  We thus need a counter that can count to whatever number fo bits we want.  

\Code{Verilog code for counting a parameterized number of bits.}{code:count}{counter.v}{verilog}

To show this off, we will have a slow count displaying, with say a count around a second or so.  This is about 50, which is close enough to 6 bits for our use.  We will thus first make a 26 bit counter, triggering the rotation of the anodes off bits 18 and 19, and the next count off bit 25.  We will thus make one more counter, this time with 16 bits so we can drive all four segments.  We just need to change our top level module and ucf to reflect this.

\Code{Verilog code for pass through and sseg counter.}{code:pass_through}{pass_through.v}{verilog} 