\chapter{Conditional Sum Adders}

\section{Introduction}



\subsection{Conditional Sum}
Conditional sum is a divide and conquer algorithm, and hence exploits binary tree parallelism.  The algorithm works by calculating both possible results for each bit (if carry in was 1 or 0), then performing paired conditional concatenation using the actual carry bit of the lower number, see Figure~\ref{f-cond_sum_add}.

\begin{figure}
\caption{Conditional Sum Adder (above), and its sub-blocks (below, left and right).}\label{f-cond_sum_add}
\begin{center}
\includegraphics{conditional_sum.png}\\\vspace{.2in}
\includegraphics{cs_4out.png} \hspace{.2in} \includegraphics{cs_2out.png}
\end{center}
\end{figure}

\begin{enumerate}
    \item form conditional terms for each digit in summation $\rightarrow$ (digit with carry, digit without carry) = ($x_i+y_i+1$,$x_i+y_i$)
    \item group by twos from right and for both conditional values in the right parenthesis form the result as follows:
    \begin{enumerate}
        \item the leftmost bit of the two terms on the right are the carry bits used to select the term on the left
        \item concatenate the appropriate term on the left (picked by carry) with each term on right after removing the parity bits of the right terms
    \end{enumerate}
    \item continue pairings until only 1 term remains. pick right number if $c_{in}=0$ else pick left.
\end{enumerate}

\begin{example}
 Add $x=0110$ and $y=1111$ by conditional sum and indicate if overflow occurred.

        {\color{ans}
        \begin{tabular}{cccc}
          0+1 & 1+1 & 1+1 & 0+1 \\
          $\downarrow$ & $\downarrow$ & $\downarrow$ & $\downarrow$ \\
          (10,01) & (11,10) & (11,10) & (10,01) \\
          $\searrow$ & $\swarrow$ & $\searrow$ & $\swarrow$ \\
          \multicolumn{2}{r}{(101,100} & \multicolumn{2}{l}{(110,101)} \\
          \multicolumn{2}{c}{$\searrow$} & \multicolumn{2}{c}{$\swarrow$} \\
          \multicolumn{4}{c}{(10110,10101)} \\
          \multicolumn{4}{c}{1 0101} \\
        \end{tabular}

        No overflow occurred (added a positive and negative number).
        }
\end{example}



\begin{example}
    Calculate $7-8$ by conditional sum.

    {\color{ans}

    $7=0111$ and $-8=1000$

    \begin{tabular}{cccc}
    $\;$ 0       & 1          & 1          & 1          \\
      +1         & 0          & 0          & 0          \\ \hline
      (10,01)    & (10,01)    & (10,01)    & (10,01)    \\
      $\searrow$ & $\swarrow$ & $\searrow$ & $\swarrow$ \\
      \multicolumn{2}{c}{(100,011)} & \multicolumn{2}{c}{(100,011)} \\
      \multicolumn{2}{c}{$\quad\searrow$} & \multicolumn{2}{c}{$\swarrow\quad$} \\
      \multicolumn{4}{c}{(10000,01111)} \\
    \end{tabular}

    Since this was done as addition no carry-in was set so the solution is \begin{tabular}{r|l} 0 & 1111 \\ \hline \end{tabular} or $-1$ in signed base ten.

    }
\end{example}


\begin{example}
Add by conditional sum $x=01100110$ and $y=00110011$.

{\color{ans}
\noindent
\begin{tabular}{rrrrrrrr}
$0+0$ & $1+0$ & $1+1$ & $0+1$ & $0+0$ & $1+0$ & $1+1$ & $0+1$ \\
$\downarrow$ & $\downarrow$ & $\downarrow$ & $\downarrow$ & $\downarrow$ & $\downarrow$ & $\downarrow$ & $\downarrow$ \\
$(01,00)$ & $(10,01)$ & $(11,10)$ & $(10,01)$ & $(01,00)$ & $(10,01)$ & $(11,10)$ & $(10,01)$ \\
$\searrow$ & $\downarrow$ & $\searrow$ & $\downarrow$ & $\searrow$ & $\downarrow$ & $\searrow$ & $\downarrow$ \\
\multicolumn{2}{r}{$(010,001)$} & \multicolumn{2}{r}{$(110,101)$} & \multicolumn{2}{r}{$(010,001)$} & \multicolumn{2}{r}{$(110,101)$} \\
& $\searrow$ & & $\swarrow$ & & $\searrow$ & & $\swarrow$  \\
&  & \multicolumn{2}{c}{$(01010,01001)$} &  &  & \multicolumn{2}{c}{$(01010,01001)$ }  \\
&  &  & $\searrow$ &  & $\swarrow$ &  &  \\
&  &  & \multicolumn{3}{c}{$(010011010,010011001)$} &  &  \\
&  &  & \multicolumn{3}{c}{$0 \, 10011001$} &  &  \\
\end{tabular}
}
\end{example}

\section{Complexity}

Why go through this?  First, by a folk theorem of Dr. Alan Laub, ``\emph{What is hard for us tends to be easy for computers (and vice versa)}.''  In reality this process is really easy for a computer to do.  Second, the process is highly parallel, so it can be done very fast.  If the numbers to be added are n bits long this takes $2(\log_2(n)+1)$ levels of logic, much better than the $n+1$ levels of logic required by ripple calculations.  Thus it is O(log(n)) in time complexity.  For example, for adding the 32 bit numbers considered already, conditional sum would take $2(\log_2(32)+1)=12$ levels of logic, so on the fast logic described it would be 12ns, a huge improvement.



\section{Recursive Build}

Keeping track of everything in this tree structure is very challenging and thus error prone. We will therefore develop a recursive description, which greatly simplifies tree based algorithms. 

Before we start, define $BT = B \mbox{\textasciicircum} c_{in}$. This can be done in one assign statement, since $B$ has $N$ bits and $c_{in}$ only has one, and you didn't use the reducing form, Verilog will conclude you meant to \textbf{xor} $c_{in}$ with each bit of $B$ separately.

As Shifu said, ``There is now a level zero.'' For us that level is adding or subtracting 2 bits without a
carry coming in. We will build a special version of a full adder for this, that outputs the carry and sum of
$A_i$ and $B_i$ and those bits $+1$. Call it CSFA (conditiional sum full adder). In the top level conditional sum
adder, we will have a generate statement and this will be our basis clause.

We now just have to write the inductive clause, which for us will have two parts. First we need to
instantiate two copies of conditional sum adder with half the bits each, to handle the upper and lower
parts of A and B. We will then instantiate two muxes for combining the four outputs of the two half-sized
conditional summer adders. Combining is done based on the carry values generated for each of two outputs
on the smaller conditional sum adder.

The final step is to use carry in to select which of the final two conditional sum adds is correct. This is
just another mux, and technically completes the 2's complement or not 2's complement.

\section{Assignment}



\begin{enumerate}
\item You must build the units above (upload .sv files)
   \begin{enumerate}
   \item Undergrads and Grads: for a 16-bit conditional sum adder.
   \item Grads: do one of the extra options we discussed in class.
   \end{enumerate}
\item Write a testbench using tasks and test your units from above. (upload sv files)
\item Program your board and take a picture of it working.
\item  Create a report that describes the project, design rational, testing rational, and results (screenshots and pictures of board running the code). (upload pdf)
\item Make sure to include pictures of your timing diagram (from test) and the picture of the functioning board.
\end{enumerate} 