\documentclass{book}
\usepackage{graphicx} % new way of doing eps files
\usepackage{rotating} % for sideways
\usepackage{multirow} % for \multirow
\usepackage{url}      % for \url
\usepackage{hyperref} % for \href
\usepackage{listings} % nice code layout
\usepackage[usenames]{color} % color
\definecolor{listinggray}{gray}{0.9}
\definecolor{graphgray}{gray}{0.7}
\definecolor{ans}{rgb}{1,0,0}
\definecolor{blue}{rgb}{0,0,1}
% \Code{title}{label}{file}{language}
\newcommand{\Code}[4]{
  \lstset{language={#4}}
  \lstset{backgroundcolor=\color{listinggray},rulecolor=\color{blue}}
  \lstset{linewidth=\textwidth}
  \lstset{commentstyle=\textit, stringstyle=\upshape,showspaces=false}
  \lstset{frame=tb}
  \lstinputlisting[caption={#1},label={#2}]{#3}
}

% source: http://stackoverflow.com/questions/741985/latex-source-code-listing-like-in-professional-books
% \SourceCode{language}{caption}{label}{file_path}
\newcommand{\SourceCode}[4]{
  % #1 = language, usually first letter is in upper case
  % #2 = caption
  % #3 = label
  % #4 = path
  \lstset{language=#1}
  \lstset{
    commentstyle=\color{orange}\textit,
    basicstyle=\footnotesize\ttfamily, % Standardschrift
    % numbers=left,               % Ort der Zeilennummern
    numberstyle=\tiny,          % Stil der Zeilennummern
    % stepnumber=2,               % Abstand zwischen den Zeilennummern
    numbersep=5pt,              % Abstand der Nummern zum Text
    tabsize=2,                  % Groesse von Tabs
    extendedchars=true,         %
    breaklines=true,            % Zeilen werden Umgebrochen
    keywordstyle=\color{red},
    frame=b,
    % keywordstyle=[1]\textbf,    % Stil der Keywords
    % keywordstyle=[2]\textbf,    %
    % keywordstyle=[3]\textbf,    %
    % keywordstyle=[4]\textbf,   \sqrt{\sqrt{}} %
    stringstyle=\color{blue}\ttfamily, % Farbe der String
    showspaces=false,           % Leerzeichen anzeigen ?
    showtabs=false,             % Tabs anzeigen ?
    xleftmargin=17pt,
    framexleftmargin=17pt,
    framexrightmargin=5pt,
    framexbottommargin=4pt,
    backgroundcolor=\color{listinggray},
    % backgroundcolor=\color{lightgray},
    showstringspaces=false      % Leerzeichen in Strings anzeigen ?
  }
  \lstinputlisting[caption={#2},label={#3}]{#4}
}

\usepackage{floatflt}
%\usepackage{cscilabs}

\topmargin 0in
\textheight 8.5in
\textwidth 6.5in
\evensidemargin 0in
\oddsidemargin 0in

\renewcommand{\chaptername}{Lab}


\title{
{\Huge Baylor University} \\
\vspace{1in}
{\large Department of} \\
{\Large Electrical and Computer Engineering}\\
\vspace{1in}
{\Large ELC 5311 (graduates), 4396(undergrad)} \\
{\Huge Advanced Digital Logic Laboratories} \\
}
\author{
Keith Schubert\\
Associate Professor\\
Department of Electrical and Computer Engineering\\
Baylor University
}
\date{}

\makeindex

\begin{document}

\baselineskip=1.05\normalbaselineskip

\maketitle

\tableofcontents

%\listoffigures

%\listoftables

\pagenumbering{arabic}

\chapter{Nexys 4 DDR Programming}

\section{Project Setup}

\begin{tabular}{l@{:}c}
Family & Artix-7\\
Sub-Family & a1000t\\
Package & csg324\\
Speed Grade & -1\\
\end{tabular}

Select ``xc7a100tcsg324-1''



\section{Passthrough}

We are going to begin with a simple project to turn on LEDs when the switch under them is on.  There are eight switches (called sw$\langle 7\rangle$ \ldots sw$\langle 0\rangle$), and eight LEDs  (called Led$\langle 7\rangle$ \ldots Led$\langle 0\rangle$).  Our first easy part will be to assign the LEDs to be identical to the switches, see Code~\ref{code:passthroughsimple}.

\Code{Verilog code for pass-through}{code:passthroughsimple}{../labs/lab_01/pass_through_simple.v}{Verilog}

This code simply passes the switches through to the LEDs, but it demonstrates the \textbf{assign} statement, which is one of our basic ways of designing combinational circuits.  The other important thing is to have a Xlilinx Design Constraints file (XDC) file.  Note before vivado (Xilinx products before version 7 of their FPGAs, i.e. Spartan-6 and below) they used a different file type called a user constraints file (UCF).  The big difference is in how timing is handled, which we will get into later in the course, for now I just want you to be aware that there are two standards and XDC should be used on FPGA families with a 7 or later.  The book will use a UCF because it is working on a Spartan-3 board.  An example of an XDC file is below.

\Code{Xlilinx Design Constraints file (XDC)}{code:XDCpassthroughsimple}{../code/common/Nexys4DDR_Master.xdc}{tcl}

\section{Making a Counter for the Seven Segment Display}

Now we want to drive the seven segment display.  All the cathodes of the LEDs in the same position are hooked together.  So for example, all four of the top LEDs' cathodes are hooked together, as are all the cathodes of the upper right, and so on.  The top LED is called A or seg$\langle 0\rangle$, the letters and numbers proceed clockwise, and the middle segment is G or seg$\langle 6\rangle$.  Each digit also has a decimal point, which is called dp, and their cathodes are also connected.   The FPGA does sink the current in this case (\emph{Why can it sink what it couldn't source?}), so a zero turns the LED on if the anode was also sourced.

To drive the seven segment display we will then need to convert from a 4 bit binary number to the seven segment display cathode pattern.  We will leave the decimal point off.  This can be seen in Code~\ref{code:sseg}.

\Code{Verilog code for converting 4 bit binary to sseg.}{code:sseg}{../labs/lab_01/sseg_driver.v}{verilog}

All the anodes for a digit are hooked together, so since there are four digits there are four anode lines (called an$\langle 3\rangle$ \ldots an$\langle 0\rangle$).  The output of the fpga is not strong enough to supply all the current for the LEDs, so a pnp transistor is used to switch power (\emph{Why does a pnp switch power better, and an npn switch ground better?}).  This means a zero, is needed to turn the pnp on and a 1 will turn it off.

This slicing allows control of 32 LEDs with 12 wires and one fourth can be on simultaneously.  Since each LED will be off three fourths of the time, we need to switch them faster than the eye can see, and since the diode's pn junction holds charge for a bit (discharges like a capacitor, why?) and thus continues to glow, and the eye continues to see light for a bit.  The Nexys 4 board has a 100MHz clock, so we need to count to about 2 million, which is 21 bits.  We thus need a counter that can count to whatever number of bits we want.

\Code{Verilog code for counting a parameterized number of bits.}{code:count}{../labs/lab_01/counter.v}{verilog}

To show this off, we will have a slow count displaying, with say a count around a second or so.  This is about 50, which is close enough to 6 bits for our use.  We will thus first make a 27 bit counter, triggering the rotation of the anodes off bits 18 and 19, and the next count off bit 26.  We will thus make one more counter, this time with 32 bits so we can drive all 8 segments.  We just need to change our top level module and xdc to reflect this.



\appendix

\chapter{Github}

Git is a version control system originally developed by Linus Torvalds to do the Linux OS development.  Github is a git web service that I will be using to develop the labs.  You can get the manual and code from:

\url{https://github.com/KeithEvanSchubert/Advanced_Digital_Logic.git}



\end{document}
